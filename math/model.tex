\documentclass[12pt]{article}


\usepackage{amsmath}   
\usepackage{amssymb}  
\usepackage{geometry}  
\geometry{margin=1in}

\begin{document}

\section{Model Formulation}

We formulate India’s long-term electricity planning problem as a cooperative planning game, where multiple generation technologies jointly determine an optimal electricity mix for the year 2047.

\subsection{Players}

Let the set of technology players be
\[
\mathcal{P} = \{N, S, W, H, C\},
\]
where:
\begin{itemize}
    \item $N$ denotes nuclear energy,
    \item $S$ denotes solar power,
    \item $W$ denotes wind power,
    \item $H$ denotes hydropower,
    \item $C$ denotes coal with carbon capture and storage (CCUS).
\end{itemize}

Each player represents a generation technology rather than an individual firm or market participant.

\subsection{Decision Variables}

Each technology $i \in \mathcal{P}$ is associated with a non-negative decision variable
\[
x_i \ge 0,
\]
where $x_i$ represents the share of total electricity generation supplied by technology $i$ in 2047.

\subsection{Adequacy Constraint}

Total electricity demand must be fully met. This is enforced through the adequacy constraint:
\[
\sum_{i \in \mathcal{P}} x_i = 1.
\]

Adequacy is treated as a hard feasibility constraint rather than a component of the payoff.

\subsection{System-Level Performance Components}

We define three system-level performance measures as linear functions of generation shares:
\begin{align}
A(x) &= \sum_{i \in \mathcal{P}} a_i x_i, \\
C(x) &= \sum_{i \in \mathcal{P}} c_i x_i, \\
R(x) &= \sum_{i \in \mathcal{P}} r_i x_i,
\end{align}
where:
\begin{itemize}
    \item $A(x)$ represents affordability,
    \item $C(x)$ represents cleanliness (emissions alignment),
    \item $R(x)$ represents reliability,
\end{itemize}
and $a_i$, $c_i$, and $r_i$ denote technology-specific contributions to these system objectives.

\subsection{Penalty Terms}

To capture system stresses and limitations associated with high penetration of certain technologies, we introduce technology-specific penalty terms:
\begin{itemize}
    \item $P_S(x_S)$: intermittency-related penalty for solar power,
    \item $P_N(x_N)$: cost and institutional pressure associated with nuclear expansion,
    \item $P_C(x_C)$: emissions and transition-related penalty for coal with CCUS.
\end{itemize}

These penalties represent endogenous system costs rather than policy preferences.

\subsection{Cooperative Welfare Objective}

The cooperative objective is to maximize national energy welfare, defined as:
\[
\max_{x} \; W(x)
= \alpha A(x) + \beta C(x) + \gamma R(x)
- P_S(x_S) - P_N(x_N) - P_C(x_C),
\]
subject to:
\[
\sum_{i \in \mathcal{P}} x_i = 1,
\quad x_i \ge 0 \;\; \forall i \in \mathcal{P}.
\]

The weights $\alpha$, $\beta$, and $\gamma$ represent system-level policy priorities for affordability, cleanliness, and reliability.

This formulation represents a cooperative planning model rather than a strategic equilibrium among competing players.

\end{document}
